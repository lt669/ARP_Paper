\section{Listening Tests}
	% Describe the listening test process

	Two rounds of listening tests were conducted for viewing position A (test 1) and B (test 2). The procedure was identical however the data used, such as video and microphone configurations used were different for each. Participants were recruited from the University of York and Abbey Road Studios for both tests, all of who were required to have some previous experience with mixing/producing and/or spatial audio. The number of participants recruited for each test were as follows:

	\vspace{5pt}
	\begin{tabular}{l c c c}
		& UoY & Abbey Road & Total\\ 
		Test 1 & 15 & 4 & 19\\ 
		Test 2 & 29 & 9 & 38\\ 
	\end{tabular}

	\subsection{Attributes Focus Group}
		The aim of the listening test was to assess the performance of each microphone array configuration for a VR environment in terms of its spatial and timbral quality. Due to the subjectivity of such a test, a focus group was assembled with the purpose of producing a list of mutually agreeable adjectives to use to describe certain spatial and timbral attributes. The attributes chosen to use within the listening tests are shown in table\ref{table:attTable}. \\

		% Table of Spatial and Timbral attributes
		\begin{table}[h]
			\begin{tabular}{m{2.2cm} | m{0.31\textwidth}}
				\textbf{Attribute} & \textbf{Description} \\ \hline
				\multicolumn{2}{l}{\textbf{Spatial}} \\ \hline
				Locatedness & How easily you can locate a sound source within the VR environment \\
				Sense of Space & How well the space where the recording was made is perceived \\
				Externalisation & Perception of sound coming from all around your head \\
				Envelopment & Whether the sounds are perceived to originate inside of outside of the head \\ \hline
				\multicolumn{2}{l}{\textbf{Timbral}} \\ \hline
				Full & Abundance of low frequencies present \\
				Bright & Abundance of high frequencies present \\
				Flat & Lack of high and low frequencies present \\
				Rich & The mix sounds good with both high and low frequencies \\
				Realistic & The sounds heard in the VR experience are realistic (sound like real instruments) and timbral characteristics have been preserved. \\
				Loud & The perceived level sounds high
			\end{tabular}
			\caption{Table of Spatial and Timbral Attributes}
			\label{table:attTable}
		\end{table} 
			

	\subsection{Procedure}

		Using an Oculus DK2 headset and a pair of Audio Technica MH50x headphones, participants were presented with an 80 second VR sample of the recording session which included the songs intro, verse and chorus using one of the microphone arrays appropriate for the samples viewing position. Once the clip was finished participants would answer a questionnaire that was spit into two main sections. The first asked them to rate on a scale of 1 - 10 each of the spatial audio attributes listed in table \ref{table:attTable} where 1 indicated they did not experience that particular attribute well and 10 being that they experience that attribute very well. The second section asked them to select as many of the timbral attributes they felt best described the overall timbre of the clip. Participants were also asked to rate on a scale of 1 - 10 how much they enjoyed the VR experience. This procedure was repeated a number of times depending on the number of microphone arrays to be presented (8 times for test 1 and 7 times for test 2). The order in which samples were presented was randomised per participant.

		To ensure a uniform understanding of the list of attributes that were used in the questionnaire a short training exercise was conducted before each test. This involved taking the participants through each of the attributes with audio examples. 

