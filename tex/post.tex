






% =========----------	[ Space left here for distraction free mode] ----------==========%










\section{Post-Processing and Workflow Set Up} % Think of a better heading

	Once the recording session at Abbey Road Studios was complete, the recording session was assessed using 5.1 surround sound system with a PC running ProTools 12 in the media suit at the University of York. Once each audio track had been previewed, the next objective was to decide on the best take recorded during the session. Each take containined minor instrumental issues such as wrong notes and timing. In a normal audio editing situation the best bits of every take could be split and merged together. However as this would not be possible to do with the video and due to the shear amount of audio tracks making this extreamly impractical, a single take must be used. It was decided that take eight of 'Close Your Eyes' would be the finaly take used. This take was exported and condenced into a smaller project where each track was exported as a WAV file ready to be imported into reaper for Ambisonic processing.

	\subsection{Video}
		As two different 360\textdegree cameras were used during recording, two different methods of spherical video encoding were used. \textbf{[Look at has's AES/thesis for this?]}

			\subsubsection{Position A - 360\textdegree perspective}
				Writing

			\subsubsection{Position B - 180\textdegree perspective}

	\subsection{Audio}
		% - Mixing of spots
		% - Position Encoding
		% - Array processing
		% - Spot mic / array encoding
		% - Binaural decoding		
		% - Reaper project layout
		% - Ambix plugins

		% - Video encoding
		% - Head/soundfield rotation integration
		% - Different viewing angles / different encoding